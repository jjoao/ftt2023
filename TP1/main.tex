\documentclass[10pt,a4paper,notitlepage]{article}
%Mise en page
\usepackage[left=2cm, right=2cm, lines=45, top=0.8in, bottom=0.7in]{geometry}
\usepackage{fancyhdr}
\usepackage{fancybox}
\usepackage{pdfpages} 
\usepackage{listings}
\renewcommand{\headrulewidth}{1.5pt}
\renewcommand{\footrulewidth}{1.5pt}
\pagestyle{fancy}
\newcommand\Loadedframemethod{TikZ}
\usepackage[framemethod=\Loadedframemethod]{mdframed}
\usepackage{tikz}
\usetikzlibrary{calc,through,backgrounds}
\usetikzlibrary{matrix,positioning}
%Desssins geometriques
\usetikzlibrary{arrows}
\usetikzlibrary{shapes.geometric}
\usetikzlibrary{datavisualization}
\usetikzlibrary{automata} % LATEX and plain TEX
\usetikzlibrary{shapes.multipart}
\usetikzlibrary{decorations.pathmorphing} 
\usepackage{pgfplots}
\usepackage{physics}
\usepackage{titletoc}
\usepackage{mathpazo} 
\usepackage{algpseudocode}
\usepackage{algorithmicx} 
\usepackage{bohr} 
\usepackage{xlop} 
\usepackage{bbding} 
\usepackage{hyperref}
%\usepackage{minibox} 
%Ecriture arabe
\usepackage{mathdesign}
\usepackage{bbding} 
\usepackage{romande} 
\lhead{
\textbf{UNIVERSIDADE DO MINHO}
}
\rhead{\textbf{TTI}
}
\chead{\textbf{Trabalho Prático}}

\lfoot{J. J. Almeida | Tiago Barata}
\cfoot{}
\rfoot{\textbf{$\mathsf{2023-2024}$}}
%\rfoot{\textit{Pr. $\mathcal{A}$.Kaal}}
%=====================Algo setup
\algblock{If}{EndIf}
\algcblock[If]{If}{ElsIf}{EndIf}
\algcblock{If}{Else}{EndIf}
\algrenewtext{If}{\textbf{si}}
\algrenewtext{Else}{\textbf{sinon}}
\algrenewtext{EndIf}{\textbf{finsi}}
\algrenewtext{Then}{\textbf{alors}}
\algrenewtext{While}{\textbf{tant que}}
\algrenewtext{EndWhile}{\textbf{fin tant que}}
\algrenewtext{Repeat}{\textbf{r\'ep\'eter}}
\algrenewtext{Until}{\textbf{jusqu'\`a}}
\algcblockdefx[Strange]{If}{Eeee}{Oooo}
[1]{\textbf{Eeee} "#1"}
{\textbf{Wuuuups\dots}}

\algrenewcommand\algorithmicwhile{\textbf{TantQue}}
\algrenewcommand\algorithmicdo{\textbf{Faire}}
\algrenewcommand\algorithmicend{\textbf{Fin}}
\algrenewcommand\algorithmicrequire{\textbf{Variables}}
\algrenewcommand\algorithmicensure{\textbf{Constante}}% replace ensure by constante
\algblock[block]{Begin}{End}
\newcommand\algo[1]{\textbf{algorithme} #1;}
\newcommand\vars{\textbf{variables } }
\newcommand\consts{\textbf{constantes}}
\algrenewtext{Begin}{\textbf{debut}}
\algrenewtext{End}{\textbf{fin}}
%================================
%================================

\setlength{\parskip}{1cm}
\setlength{\parindent}{1cm}
\setlength{\parskip}{0mm}
\setlength{\parindent}{10mm}

\hypersetup{
    colorlinks=true,
    linkcolor=blue,
    filecolor=magenta,      
    urlcolor=cyan,
    pdftitle={Overleaf Example},
    pdfpagemode=FullScreen,
    }

\usepackage{xspace} % deteta se a seguir a palavra tem uma palavra ou um sinal de pontuaçao se tiver uma palavra da espaço, se for um sinal de pontuaçao nao da espaço

\usepackage{placeins}

\usepackage{fancyvrb}
\fvset{fontsize=\scriptsize}

%===========================================================
\begin{document}

\begin{center}
\section*{Terminologias com NATerm}
\end{center}

\subsection*{NATerm}
Para este trabalho vamos usar a ferramenta NATerm, que estamos a usar nas aulas. O NATerm é usado para criar terminologias. Aceita ficheiros com sintaxe própria e gera vários tipos de ficheiros, como \textit{PDF}, \textit{HTML} ou \textit{XDXF}. 
\\
O NATerm pode ser usada como instalação local, mas vamos usar a sua \href{https://natura.di.uminho.pt/jjbin/naterm}{versão online}.

\subsection*{Usar o NATerm}
O funcionamento do NATerm foi explorado nas aulas. Segue um exemplo de base da sintaxe do NATerm.

\subsubsection*{Cabeçalhos}
\begin{Verbatim}
    %title Exemplo NATerm
    %author JJoao; TBarata
    %date 2023-09-27
    %lang EN PT
    %inv hpr hpn
    %rellang EN
\end{Verbatim}

\subsubsection*{Entrada}
\begin{Verbatim}
    PT : gato
    +var : ptpt
    +G : m
    EN : cat
    EN : pussy-cat
    +G : m
    def : domestic feline
    !obs : são fofos
    !img : gato.jpg
\end{Verbatim}
$Nota_1:$ Os atributos de cada entrada são todos opcionais. Use o mais conveniente.
\\
$Nota_2:$ Entre cada entrada, é necessário deixar uma linha em branco.
\\
$Nota_3:$ As imagens devem ser colocadas numa pasta \texttt{MEDIA.zip} e carregada juntamente com a terminologia.

\subsection*{Enunciado}
Criar uma terminologia NATerm sobre um dos seguintes temas:
\begin{itemize}
    \item Jardinagem
    \item Bombeiros
    \item Bicicleta
    \item Pastelaria
    \item Atletismo (saltos)
\end{itemize}
O objetivo desta terminologia é ser o mais detalhada possível. Por exemplo, se a terminologia fosse sobre o corpo humano, podíamos colocar todos os ossos, todos os músculos, movimentos, etc... Os termos têm de estar escritos em \textbf{pelo menos 2 línguas, onde o português é obrigatório.}
Antes de criar a terminologia imagine que esta será criada para:
\begin{description}
    \item[Cenário 1] Estou a criar uma terminologia para um aluno de Erasmus que tem muito interesse e conhecimentos sobre o tema selecionado
    \item[Cenário 2] Estou a criar uma terminologia para uma pessoa que está a começar a aprender sobre o tema selecionado 
\end{description}

\subsection*{Entrega}
O trabalho deve ser feito em grupos com no \textbf{máximo 3 elementos}, que devem entregar o ficheiro com a extensão \texttt{.naterm} até \textbf{29 de outubro}. Posteriormente indicar-se-á o local de submissão.

\end{document}